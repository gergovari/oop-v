\section{Ismertesse ábrával a „mag”, „tartó”, „$\alpha$ vágat”, „szigorú $\alpha$ vágat” és „magasság” fogalmakat a fuzzy halmazok esetén!}

A fuzzy halmazok jellemzésére és a hagyományos (crisp) halmazokkal való összekapcsolására speciális fogalmakat használunk. Ezek a fogalmak a tagsági függvény ($\mu_A(x)$) különböző tulajdonságait írják le.

\subsection{Szemléltető Ábra}

Az alábbi ábra egy trapéz alakú fuzzy halmazon mutatja be a definíciókat.

\begin{center}
    \begin{figure}[hbt!]
		\centering
		\includegraphics[scale=.3]{./images/fuzzy.jpg}
		\caption{Fuzzy halmaz jellemzői: magasság, tartó, mag, $\alpha$-vágat és szigorú $\alpha$-vágat}
	\end{figure}
\end{center}

\subsection{Definíciók}

Legyen $X$ az alaphalmaz (univerzum) és $A$ egy ezen értelmezett fuzzy halmaz.

\begin{itemize}
    \item \textbf{Magasság (Height):}
        A tagsági függvény által felvett legnagyobb érték (szuprémum).
        $$ \text{hgt}(A) = \sup_{x \in X} \mu_A(x) $$
        \begin{itemize}
            \item Ha $\text{hgt}(A) = 1$, a fuzzy halmazt \textbf{normálnak} nevezzük.
            \item Ha $\text{hgt}(A) < 1$, a fuzzy halmaz \textbf{szubnormál}.
        \end{itemize}

    \item \textbf{Tartó (Support):}
        Az alaphalmaz azon elemeinek halmaza (hagyományos halmaz), amelyek tagsági foka \textit{nagyobb, mint nulla}.
        $$ \text{supp}(A) = \{ x \in X \mid \mu_A(x) > 0 \} $$
        Gyakorlatilag ez a fuzzy halmaz "szélessége" az alján.

    \item \textbf{Mag (Core):}
        Az alaphalmaz azon elemeinek halmaza, amelyek \textit{teljes mértékben} (1-es értékkel) tartoznak a fuzzy halmazhoz.
        $$ \text{core}(A) = \{ x \in X \mid \mu_A(x) = 1 \} $$
        
    \item \textbf{$\alpha$-vágat ($\alpha$-cut):}
        Egy $\alpha \in [0, 1]$ szinten vett vízszintes metszet. Az eredmény egy olyan hagyományos halmaz, amely tartalmazza mindazon elemeket, amelyek tagsági foka eléri vagy meghaladja az $\alpha$ szintet.
        $$ A_\alpha = \{ x \in X \mid \mu_A(x) \geq \alpha \} $$
        \textit{Megjegyzés:} A mag nem más, mint az $1$-vágat ($A_1$).

    \item \textbf{Szigorú $\alpha$-vágat (Strong $\alpha$-cut):}
        Hasonló a sima vágathoz, de itt szigorú egyenlőtlenséget használunk (az éppen $\alpha$ értékű elemek nem kerülnek bele).
        $$ A_{\bar{\alpha}} = \{ x \in X \mid \mu_A(x) > \alpha \} $$
        \textit{Megjegyzés:} A tartó nem más, mint a szigorú $0$-vágat ($A_{\bar{0}}$).
\end{itemize}
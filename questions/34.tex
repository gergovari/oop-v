\section{Ismertesse a string-numerikus adat közti konverzióra használt osztályt!}

A C++ nyelvben a karakterláncok (stringek) és numerikus típusok (int, double stb.) közötti kétirányú, formázott konverzióra leggyakrabban az \texttt{std::stringstream} osztályt használjuk. Ez az osztály a \texttt{<sstream>} header fájlban található.

\subsection{Az osztály jellemzői és felépítése}

A \texttt{stringstream} egyesíti a bemeneti és kimeneti adatfolyamok tulajdonságait, de nem konzolra vagy fájlba ír, hanem a memóriában dolgozik.

\begin{itemize}
    \item \textbf{Öröklődés:} Az \texttt{iostream} leszármazottja, így ugyanazokkal az operátorokkal (\texttt{<<}, \texttt{>>}) kezelhető, mint a \texttt{cin} vagy a \texttt{cout}.
    \item \textbf{Belső puffer:} Egy belső \texttt{std::string} objektumot kezel pufferként; ebbe írunk bele vagy ebből olvasunk ki.
    \item \textbf{Típusbiztonság:} A C-stílusú \texttt{sprintf}-fel ellentétben típusbiztos és nem okoz puffertúlcsordulást.
\end{itemize}

\subsection{Fő metódusok és kezelés}

\begin{itemize}
    \item \textbf{\texttt{str()}:}
        \begin{itemize}
            \item Paraméter nélkül: Visszaadja a belső puffer tartalmát \texttt{std::string}-ként (pl. konverzió végeredménye).
            \item Paraméterrel (string): Beállítja a belső puffer tartalmát (pl. konverzió kezdete).
        \end{itemize}
    \item \textbf{\texttt{clear()}:}
        \begin{itemize}
            \item Törli az állapotjelző biteket (pl. EOF, failbit).
            \item \textbf{Fontos:} Nem törli a tartalmat! Ha újra akarjuk használni a streamet, a tartalmat az \texttt{str("")} hívással, a hibaállapotot a \texttt{clear()} hívással kell alaphelyzetbe állítani.
        \end{itemize}
\end{itemize}

\subsection{Konverziós irányok}

\begin{itemize}
    \item \textbf{Szám $\rightarrow$ String (Szerializáció):}
        \begin{enumerate}
            \item Adat beírása a streambe a \texttt{<<} operátorral.
            \item Eredmény kinyerése az \texttt{.str()} metódussal.
        \end{enumerate}
    \item \textbf{String $\rightarrow$ Szám (Parsolás):}
        \begin{enumerate}
            \item A string betöltése a streambe (konstruktorban vagy \texttt{.str()} hívással).
            \item Adat kiolvasása célváltozóba a \texttt{>>} operátorral.
            \item A stream automatikusan kezeli a whitespace karaktereket elválasztóként.
        \end{enumerate}
\end{itemize}

\subsection{Mintapélda}

\begin{minted}{cpp}
#include <iostream>
#include <sstream> // Kötelező header
#include <string>

int main() {
    // 1. Konverzió: Szám -> String
    int szam = 42;
    double lebegopontos = 3.14;
    
    std::stringstream ss_out;
    
    // Beírás a streambe (mint a cout-nál)
    ss_out << szam << " " << lebegopontos;
    
    // Kinyerés stringként
    std::string eredmeny = ss_out.str();
    std::cout << "Stringge alakitva: " << eredmeny << std::endl;

    // -----------------------------------------

    // 2. Konverzió: String -> Szám
    std::string bemenet = "1985 75.5";
    std::stringstream ss_in(bemenet); // Inicializálás stringgel
    
    int ev;
    float suly;
    
    // Kiolvasás változókba (mint a cin-nél)
    ss_in >> ev >> suly;
    
    if (!ss_in.fail()) {
        std::cout << "Ev: " << ev << ", Suly: " << suly << std::endl;
    }

    return 0;
}
\end{minted}
\section{Ismertesse példával a fuzzy következtető módszer működését!}

A fuzzy következtetés (Inference) folyamatát a legszemléletesebben a **Mamdani-típusú** módszerrel lehet bemutatni. A példában egy egyszerű **Ventilátor Szabályozót** valósítunk meg, ahol a bemenet a hőmérséklet, a kimenet a ventilátor fordulatszáma.

\subsection{A példa paraméterei}

\begin{itemize}
    \item \textbf{Bemenet ($x$):} Hőmérséklet ($0 \dots 40^\circ\text{C}$).
    \item \textbf{Kimenet ($y$):} Fordulatszám ($0 \dots 1000 \text{ RPM}$).
    \item \textbf{Szabálybázis (2 szabály):}
        \begin{enumerate}
            \item \textbf{R1:} HA Hőmérséklet = \textit{Hideg}, AKKOR Fordulat = \textit{Lassú}.
            \item \textbf{R2:} HA Hőmérséklet = \textit{Meleg}, AKKOR Fordulat = \textit{Gyors}.
        \end{enumerate}
    \item \textbf{Aktuális mérés (Crisp input):} $x_0 = 28^\circ\text{C}$.
\end{itemize}

\subsection{1. Lépés: Fuzzifikálás (Fuzzification)}

A konkrét mért értéket ($28^\circ\text{C}$) levetítjük a bemeneti tagsági függvényekre, hogy megkapjuk a tagsági fokokat ($\mu$).

\begin{itemize}
    \item A $28^\circ\text{C}$ már inkább meleg, de kicsit még hidegnek is számíthat (az átfedés miatt).
    \item Leolvasott értékek:
        \begin{itemize}
            \item $\mu_{Hideg}(28) = 0.2$ (20\%-ban tartozik a Hideg halmazba).
            \item $\mu_{Meleg}(28) = 0.8$ (80\%-ban tartozik a Meleg halmazba).
        \end{itemize}
\end{itemize}

\subsection{2. Lépés: Kiértékelés (Inference / Implication)}

Meghatározzuk a szabályok "tüzelési erősségét" (activation strength), és ezt alkalmazzuk a kimeneti halmazokra.

\begin{itemize}
    \item \textbf{R1 szabály (Hideg $\rightarrow$ Lassú):}
        \begin{itemize}
            \item A szabály feltétele $0.2$-es erősséggel teljesült.
            \item A kimeneti \textit{Lassú} fuzzy halmazt **levágjuk (csonkoljuk)** a $0.2$-es magasságnál (Minimum operátor).
            \item Eredmény: Egy $0.2$ magas trapéz (a \textit{Lassú} halmaz alja).
        \end{itemize}
    \item \textbf{R2 szabály (Meleg $\rightarrow$ Gyors):}
        \begin{itemize}
            \item A szabály feltétele $0.8$-as erősséggel teljesült.
            \item A kimeneti \textit{Gyors} fuzzy halmazt **levágjuk** a $0.8$-as magasságnál.
            \item Eredmény: Egy $0.8$ magas trapéz (a \textit{Gyors} halmaz nagy része).
        \end{itemize}
\end{itemize}

\subsection{3. Lépés: Aggregáció (Aggregation)}

A szabályok részeredményeit egyesítjük egyetlen eredő fuzzy halmazzá.

\begin{itemize}
    \item A két csonkolt alakzatot (a kicsi \textit{Lassú}-t és a nagy \textit{Gyors}-at) "egymásra rakjuk" az **Unió (Maximum)** művelettel.
    \item Az eredmény egy komplex, szabálytalan alakzat, amely a kimeneti univerzum felett helyezkedik el.
\end{itemize}

\subsection{4. Lépés: Defuzzifikálás (Defuzzification)}

Az aggregált alakzatból egyetlen konkrét számot kell nyernünk a ventilátor vezérléséhez.

\begin{itemize}
    \item \textbf{Módszer:} Súlypontszámítás (Center of Gravity - COG).
    \item Megkeressük az eredő síkidom súlypontjának $x$ koordinátáját.
    \item Mivel a \textit{Gyors} halmaz dominál ($0.8$ vs $0.2$), a súlypont a magasabb tartomány felé tolódik.
    \item \textbf{Eredmény:} $y_{out} \approx 750 \text{ RPM}$.
\end{itemize}

\subsection{Programkód (Szimuláció)}

\begin{minted}{python}
# Pszeudokód a Mamdani következtetésre

# 1. Bemeneti mérés
x_temp = 28

# 2. Fuzzifikálás (Tagsági függvények lekérdezése)
# Feltételezve, hogy definiált függvények
mu_hideg = membership_hideg(x_temp) # 0.2
mu_meleg = membership_meleg(x_temp) # 0.8

# 3. Kiértékelés (Implikáció - MIN operátor)
# A kimeneti halmazok "levágása"
rule1_shape = min(mu_hideg, output_set_lassu) # Lassú halmaz max 0.2 magasan
rule2_shape = min(mu_meleg, output_set_gyors) # Gyors halmaz max 0.8 magasan

# 4. Aggregáció (Unió - MAX operátor)
final_fuzzy_shape = max(rule1_shape, rule2_shape)

# 5. Defuzzifikálás (Súlypont)
# Integrálás az y tengely mentén
output_rpm = center_of_gravity(final_fuzzy_shape)

print(f"Ventilator fordulat: {output_rpm} RPM")
\end{minted}
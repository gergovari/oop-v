\section{Ismertesse az osztálypéldányokon végzett műveletek definiálási lehetőségeit! Mely műveleteket nem lehet átdefiniálni?}

\subsection{Az operátor-túlterhelés (Operator Overloading) lehetőségei}
C++-ban az operátorok (pl. +, -, ==) új jelentést kaphatnak felhasználói típusok (osztályok) esetén. Két alapvető módon definiálhatjuk őket:

\begin{enumerate}
    \item \textbf{Tagfüggvényként (Member Function):}
    \begin{itemize}
        \item Az operátor az osztály része.
        \item \textbf{Bal oldali operandus:} Implicit módon mindig az aktuális objektum (\texttt{*this}).
        \item \textbf{Paraméterek száma:} Eggyel kevesebb, mint az operandusok száma (bináris operátornál 1 paraméter, unárisnál 0).
        \item \textbf{Kötelező így írni:} \texttt{=}, \texttt{[]}, \texttt{()}, \texttt{->}.
    \end{itemize}

    \item \textbf{Globális (szabad) függvényként (Global Function):}
    \begin{itemize}
        \item Az osztályon kívül definiáljuk.
        \item \textbf{Bal oldali operandus:} Az első paraméterként adjuk át (nincs \texttt{this}).
        \item \textbf{Barát (friend) státusz:} Gyakran szükséges, hogy a függvény hozzáférjen a privát adattagokhoz.
        \item \textbf{Előnye:} Lehetővé teszi a szimmetrikus konverziót (pl. \texttt{10 + obj} és \texttt{obj + 10} is működhet).
    \end{itemize}
\end{enumerate}

\subsection{Nem átdefiniálható operátorok}
A nyelv védelme érdekében bizonyos operátorok működése rögzített, ezeket \textbf{tilos} túlterhelni:

\begin{itemize}
    \item \texttt{.} (Pont operátor / Tagkiválasztás)
    \item \texttt{::} (Hatókör-feloldó / Scope resolution)
    \item \texttt{?:} (Feltételes / Ternary operátor)
    \item \texttt{sizeof} (Méret lekérdezése)
    \item \texttt{typeid} (Típusinformáció)
    \item \texttt{.*} (Tagra mutató pointer feloldása)
\end{itemize}

\subsection{Példakód}
Az alábbi példa bemutatja az összeadás (\texttt{+}) globális barátként, és az értékadás (\texttt{+=}) tagfüggvényként történő megvalósítását.

\begin{minted}[frame=lines, framesep=2mm, baselinestretch=1.2, fontsize=\footnotesize, linenos]{cpp}
#include <iostream>

class Vector2 {
private:
    int x, y;

public:
    Vector2(int x, int y) : x(x), y(y) {}

    // 1. Tagfüggvényként definiált operátor (+=)
    // A bal oldali operandus a 'this', a jobb oldali az 'other'
    // Módosítja az objektum állapotát.
    Vector2& operator+=(const Vector2& other) {
        this->x += other.x;
        this->y += other.y;
        return *this; // Láncolhatóság miatt referenciával térünk vissza
    }

    // 2. Globális (Barát) függvényként definiált operátor (+)
    // Két paramétert kap, új objektumot hoz létre.
    friend Vector2 operator+(const Vector2& lhs, const Vector2& rhs);

    void print() const { std::cout << x << "," << y << std::endl; }
};

// Globális definíció
Vector2 operator+(const Vector2& lhs, const Vector2& rhs) {
    // Nem módosítjuk a paramétereket, új példányt adunk vissza
    return Vector2(lhs.x + rhs.x, lhs.y + rhs.y);
}

int main() {
    Vector2 v1(1, 2);
    Vector2 v2(3, 4);

    Vector2 v3 = v1 + v2; // Globális operator+ hívása
    v1 += v2;             // Tagfüggvény operator+= hívása

    v3.print(); // 4,6
    v1.print(); // 4,6
    return 0;
}
\end{minted}
\section{Ismertesse a Zadeh szerinti „s-norma”, „t-norma” és komplemens képzését fuzzy halmazoknál!}

A fuzzy logikában a hagyományos halmazműveletek (komplemens, metszet, unió) általánosításra kerülnek, hogy értelmezhetők legyenek a $[0, 1]$ intervallumon mozgó tagsági értékekre. Lotfi A. Zadeh eredeti definíciói alkotják a „standard” fuzzy műveleteket.

\subsection{1. Fuzzy Komplemens (Tagadás)}

A klasszikus logikai „NEM” ($\neg$) művelet kiterjesztése. Zadeh az úgynevezett \textbf{standard komplemenst} vezette be.

\begin{itemize}
    \item \textbf{Jelölése:} $\overline{A}$ vagy $\neg A$.
    \item \textbf{Működési elv:} Minél inkább tagja egy elem a halmaznak, annál kevésbé tagja a komplemensének (szimmetria a 0.5 értékre).
    \item \textbf{Képlet:}
    $$ \mu_{\overline{A}}(x) = 1 - \mu_A(x) $$
    \item \textbf{Axiómák (követelmények):}
        \begin{itemize}
            \item \textit{Peremfeltételek:} $c(0)=1$ és $c(1)=0$.
            \item \textit{Monotonitás:} Ha $\mu_A(x)$ nő, a komplemens csökken.
            \item \textit{Involúció:} A tagadás tagadása az eredeti érték ($\neg(\neg x) = x$).
        \end{itemize}
\end{itemize}

\subsection{2. T-norma (Fuzzy Metszet / ÉS)}

A „t-norma” (Triangular norm) a klasszikus „ÉS” (metszet, $\cap$) művelet általánosított, axiomatikus leírása.

\begin{itemize}
    \item \textbf{Zadeh definíciója (Standard metszet):}
        Zadeh a \textbf{MINIMUM} operátort javasolta a fuzzy metszet képzésére.
        $$ \mu_{A \cap B}(x) = \min(\mu_A(x), \mu_B(x)) $$
        Ez a létező legnagyobb (legszigorúbb) t-norma.
        
    \item \textbf{Szemléletes jelentése:} Egy lánc olyan erős, mint a leggyengébb láncszeme. Ha két feltételnek egyszerre kell teljesülnie, az eredő igazságértéket a kevésbé teljesülő határozza meg.
    
    \item \textbf{Általános T-norma axiómák:}
        Egy $T(x, y): [0,1]^2 \rightarrow [0,1]$ függvény t-norma, ha:
        \begin{itemize}
            \item \textit{Kommutatív:} $T(a, b) = T(b, a)$
            \item \textit{Asszociatív:} $T(a, T(b, c)) = T(T(a, b), c)$
            \item \textit{Monoton növekvő:} Ha a bemenetek nőnek, az eredmény nem csökkenhet.
            \item \textit{Peremfeltétel (1 az egységelem):} $T(a, 1) = a$
        \end{itemize}
\end{itemize}

\subsection{3. S-norma (Fuzzy Unió / VAGY)}

Az „s-norma” (Triangular conorm / T-conorm) a klasszikus „VAGY” (unió, $\cup$) művelet általánosítása.

\begin{itemize}
    \item \textbf{Zadeh definíciója (Standard unió):}
        Zadeh a \textbf{MAXIMUM} operátort javasolta a fuzzy unió képzésére.
        $$ \mu_{A \cup B}(x) = \max(\mu_A(x), \mu_B(x)) $$
        Ez a létező legkisebb s-norma.

    \item \textbf{Szemléletes jelentése:} Ha két feltétel közül elég az egyiknek teljesülnie, az eredő igazságértéket a jobban teljesülő határozza meg.

    \item \textbf{Általános S-norma axiómák:}
        Egy $S(x, y)$ függvény s-norma, ha kommutatív, asszociatív, monoton, és:
        \begin{itemize}
            \item \textit{Peremfeltétel (0 az egységelem):} $S(a, 0) = a$
        \end{itemize}
\end{itemize}

\subsection{Összefüggés: De Morgan azonosságok}

A Zadeh-féle operátorok (Min, Max, $1-x$) konzisztensek egymással, azaz teljesítik a De Morgan törvényeket, ahogy a klasszikus logika is:

$$ \neg(A \cap B) = \neg A \cup \neg B $$
$$ 1 - \min(a, b) = \max(1-a, 1-b) $$

\subsection{Implementációs példa}

Az alábbi kód bemutatja a Zadeh-féle operátorok működését:

\begin{minted}{python}
def fuzzy_operations_zadeh(mu_A, mu_B):
    # 1. Komplemens (NOT)
    not_A = 1.0 - mu_A
    
    # 2. T-norma (Metszet / AND) -> MINIMUM
    intersection = min(mu_A, mu_B)
    
    # 3. S-norma (Unió / OR) -> MAXIMUM
    union = max(mu_A, mu_B)
    
    return not_A, intersection, union

# Példa értékek
a = 0.7
b = 0.4

res = fuzzy_operations_zadeh(a, b)
# Eredmény:
# NOT A: 0.3
# AND:   0.4 (mert 0.4 < 0.7)
# OR:    0.7 (mert 0.7 > 0.4)
\end{minted}
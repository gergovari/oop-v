\section{Ismertesse a C++ nyelvben a függvények alapértelmezett paraméterezésének lehetőségét, és ennek szabályait!}

\subsection{Fogalma és Célja}
\begin{itemize}
    \item \textbf{Definíció:} Lehetőség arra, hogy a függvény paramétereinek előre megadott (default) értéket rendeljünk.
    \item \textbf{Működés:} Ha a függvényhívás során a hívó fél nem ad meg argumentumot az adott paraméterhez, a fordító automatikusan az alapértelmezett értéket illeszti be.
    \item \textbf{Haszna:}
    \begin{itemize}
        \item Függvénytúlterhelés (function overloading) egyszerűsítése vagy kiváltása.
        \item A kód olvashatóságának növelése.
        \item Meglévő függvények bővítése új paraméterekkel a meglévő hívások "eltörése" nélkül.
    \end{itemize}
\end{itemize}

\subsection{Szintaxis}
Az alapértelmezett értéket az értékadó operátorral (\texttt{=}) adjuk meg a paraméter típusát és nevét követően.

\begin{minted}[frame=lines, framesep=2mm, baselinestretch=1.2, fontsize=\footnotesize, linenos]{cpp}
// Deklaráció (prototípus)
void ablakNyit(int szelesseg, int magassag, bool teljesKepernyo = false);

// Hívások
ablakNyit(800, 600);        // teljesKepernyo = false (alapértelmezett)
ablakNyit(1920, 1080, true); // teljesKepernyo = true (felülírt)
\end{minted}

\subsection{Alapvető Szabályok}
Az alábbi szabályok ismerete elengedhetetlen a helyes használathoz:

\begin{enumerate}
    \item \textbf{Jobbról-balra szabály (Right-to-left rule):}
    Ha egy paraméternek alapértelmezett értéket adunk, akkor az \textbf{összes} utána következő (tőle jobbra lévő) paraméternek is rendelkeznie kell alapértelmezett értékkel.
    
    \item \textbf{Deklaráció vs. Definíció:}
    Az alapértelmezett értékeket általában a függvény \textbf{deklarációjában} (header fájlban, prototípusban) adjuk meg.
    \begin{itemize}
        \item Ha a definíció (implementáció) külön van, ott \textbf{TILOS} megismételni az alapértékeket (fordítási hiba, még akkor is, ha ugyanazt az értéket írnánk be).
    \end{itemize}
    
    \item \textbf{Argumentumok elhagyása:}
    Argumentumokat csak a paraméterlista \textbf{végéről} hagyhatunk el. Nem lehet "lyukasan" paraméterezni (pl. az elsőt és harmadikat megadjuk, de a középsőt az alapértékre bízzuk).
\end{enumerate}

\newpage
\subsection{Kódpélda a szabályokra}

\begin{minted}[frame=lines, framesep=2mm, baselinestretch=1.2, fontsize=\footnotesize, linenos]{cpp}
// HELYES: Jobbról balra haladva mindenki kapott értéket
void teszt(int a, int b = 5, int c = 10);

// HELYTELEN: 'b'-nek van default értéke, de a tőle jobbra lévő 'c'-nek nincs
// void hibas(int a, int b = 5, int c); // -> Fordítási hiba!

// Implementáció: Itt már NEM szerepelnek az egyenlőségjelek
void teszt(int a, int b, int c) {
    // ... kód ...
}

int main() {
    teszt(1);       // a=1, b=5, c=10
    teszt(1, 2);    // a=1, b=2, c=10
    teszt(1, 2, 3); // a=1, b=2, c=3
    
    // teszt(1, , 3); // -> Fordítási hiba! (nem lehet kihagyni a közepét)
    return 0;
}
\end{minted}
\section{Ismertesse a tisztán virtuális metódus készítésének szintaktikáját! Hogyan nevezzük a legalább 1 tisztán virtuális metódust tartalmazó osztályt? Milyen szabályok vonatkoznak erre az osztályra?}

A C++ nyelvben a tisztán virtuális (pure virtual) metódusok szolgálnak arra, hogy egy osztályban csak a függvény interfészét (szignatúráját) határozzuk meg, a megvalósítást (implementációt) pedig kötelezően a leszármazottakra bízzuk.

\subsection{Szintaktika}

A tisztán virtuális metódust a deklaráció végére írt \textbf{\texttt{= 0}} jelöléssel (pure-specifier) hozzuk létre.

\begin{itemize}
    \item A függvénynek általában nincs törzse (implementációja) az adott osztályban.
    \item A \texttt{virtual} kulcsszó használata kötelező.
\end{itemize}

\begin{minted}{cpp}
class Alakzat {
public:
    // Tisztán virtuális metódus
    // A "= 0" jelzi, hogy itt nincs implementáció
    virtual double terulet() const = 0; 
    
    // Virtuális destruktor (ajánlott)
    virtual ~Alakzat() {}
};
\end{minted}

\subsection{Elnevezés}

Azt az osztályt, amely legalább egy tisztán virtuális metódust tartalmaz, \textbf{absztrakt osztálynak} (Abstract Class) nevezzük.
Gyakran használják őket interfészként (Interface), ahol az összes metódus tisztán virtuális.

\subsection{Az absztrakt osztályra vonatkozó szabályok}

Az absztrakt osztályok viselkedése eltér a hagyományos (konkrét) osztályokétól:

\begin{itemize}
    \item \textbf{Példányosítás tilalma:}
        Absztrakt osztályból közvetlenül \textbf{nem hozható létre objektum} (példány).
        \begin{minted}{cpp}
Alakzat a; // HIBA: absztrakt osztály nem példányosítható
        \end{minted}
    
    \item \textbf{Pointerek és Referenciák:}
        Bár példány nem hozható létre, \textbf{mutató (pointer) vagy referencia} típusként használható. Ez teszi lehetővé a polimorfizmust.
        \begin{minted}{cpp}
Alakzat* mutato = new Kor(); // HELYES (ha a Kor konkrét)
        \end{minted}

    \item \textbf{Leszármaztatási kötelezettség:}
        Ha egy leszármazott osztály nem definiálja felül (override) az összes örökölt tisztán virtuális metódust (nem ad nekik törzset), akkor a leszármazott osztály is \textbf{absztrakt marad}, és nem lehet példányosítani.
        
    \item \textbf{Adattagok és konkrét metódusok:}
        Az absztrakt osztály tartalmazhat normál (nem tisztán virtuális) tagfüggvényeket és adattagokat is, amelyek a közös logikát valósítják meg.
\end{itemize}
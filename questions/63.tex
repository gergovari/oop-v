\section{Ismertesse, hogy mikor alkalmazható egy t-norma, s-norma, komplemens definicióit tartalmazó szabályrendszer fuzzy halmazműveletekhez! Mit alkotnak ilyenkor a szabályrendszer elemei?}

A fuzzy logikában a különböző műveleteket (metszet, unió, negáció) nem választhatjuk meg egymástól függetlenül, ha konzisztens matematikai rendszert szeretnénk építeni. A három operátornak összhangban kell lennie egymással.

\subsection{Alkalmazhatóság feltétele: A De Morgan Hármas}

Egy t-norma ($T$), s-norma ($S$) és komplemens ($c$) definícióit tartalmazó szabályrendszer akkor alkalmazható helyesen fuzzy halmazműveletekhez, ha azok kielégítik a \textbf{dualitás elvét}, azaz egymás duális párjai a megadott negációra nézve.

Ezt a rendszert \textbf{De Morgan Hármasnak} (De Morgan Triplet) nevezzük.

A rendszer akkor konzisztens, ha teljesülnek az általánosított De Morgan azonosságok minden $x, y \in [0, 1]$ esetén:

$$ c(S(x, y)) = T(c(x), c(y)) $$
$$ c(T(x, y)) = S(c(x), c(y)) $$

Ha a standard komplemenst használjuk ($c(x) = 1-x$), akkor a feltétel egyszerűsödik:
$$ 1 - S(x, y) = T(1-x, 1-y) $$

\textbf{Példák érvényes hármasokra:}
\begin{itemize}
    \item \textbf{Zadeh (Standard):} Min, Max, $1-x$.
    \item \textbf{Szorzat (Probabilisztikus):} $xy$, $x+y-xy$, $1-x$.
    \item \textbf{Lukasiewicz (Korlátos):} $\max(0, x+y-1)$, $\min(1, x+y)$, $1-x$.
\end{itemize}

\subsection{Mit alkotnak az elemek?}

Ha a fenti feltételek teljesülnek, a $([0, 1], T, S, c)$ struktúra egy \textbf{De Morgan Algebrát} alkot.

Ez az algebra hasonlít a klasszikus Boole-algebrára, de \textbf{két fontos axióma NEM teljesül} benne általánosan:

\begin{enumerate}
    \item \textbf{A harmadik kizárásának elve (Law of Excluded Middle):}
        A klasszikus logikában $A \cup \neg A = 1$. A fuzzy logikában általában:
        $$ S(x, c(x)) \neq 1 $$
        (Pl. Zadeh operátoroknál $0.5 \cup 0.5 = 0.5$).
    \item \textbf{Az ellentmondásmentesség elve (Law of Contradiction):}
        A klasszikus logikában $A \cap \neg A = 0$. A fuzzy logikában általában:
        $$ T(x, c(x)) \neq 0 $$
\end{enumerate}

\textbf{Következmény:} A fuzzy logika matematikailag egy disztributív (általában), De Morgan algebra a $[0, 1]$ intervallumon, amely a crisp (0 vagy 1) értékek esetén Boole-algebrává degenerálódik.

\begin{minted}{cpp}
// Példa: A Lukasiewicz hármas implementációja (De Morgan Triplet)
#include <algorithm>
#include <iostream>

// Komplemens: 1 - x
float c(float x) { return 1.0f - x; }

// T-norma (Lukasiewicz): max(0, a + b - 1)
float t_norm(float a, float b) { return std::max(0.0f, a + b - 1.0f); }

// S-norma (Lukasiewicz): min(1, a + b)
float s_norm(float a, float b) { return std::min(1.0f, a + b); }

bool check_de_morgan(float a, float b) {
    // Ellenőrzés: NOT (A OR B) == (NOT A) AND (NOT B)
    float left_side = c(s_norm(a, b));
    float right_side = t_norm(c(a), c(b));
    
    // Lebegőpontos összehasonlításnál epszilon kellene, de elvben egyenlőek
    return left_side == right_side; 
}
\end{minted}
\section{Ismertesse ábrával a „közvetlen bázisosztály” és a „közvetett bázisosztály” fogalmakat!}

Az objektumorientált programozásban az öröklődési hierarchia mélysége alapján különböztetjük meg az ősöket. Ez a megkülönböztetés fontos a névütközések feloldása, a konverziók és a konstruktorhívások szempontjából.

\subsection{Fogalmak definíciója}

\begin{itemize}
    \item \textbf{Közvetlen bázisosztály (Direct Base Class):}
        Az az osztály, amelyből az adott osztályt specifikusan származtattuk. Ez az osztály szerepel a leszármazott osztály definíciójában a kettőspont után.
    \item \textbf{Közvetett bázisosztály (Indirect Base Class):}
        Az öröklődési láncban feljebb (távolabb) elhelyezkedő ősök (pl. a „nagyszülő”). Ezek tulajdonságait a leszármazott a köztes osztályokon keresztül, tranzitív módon örökli.
\end{itemize}

\subsection{Strukturális Ábra}

Az alábbi diagram a \textbf{„C” osztály szemszögéből} mutatja be a relációkat egy többszintű öröklődés (Multilevel Inheritance) esetén.



\begin{verbatim}
      +-------------------+
      |     Osztály A     | <------- "C" KÖZVETETT bázisosztálya
      +-------------------+          (A "B"-nek közvetlen őse)
                ^
                |
      +---------+---------+
      |     Osztály B     | <------- "C" KÖZVETLEN bázisosztálya
      +-------------------+          (A definícióban szerepel)
                ^
                |
      +---------+---------+
      |     Osztály C     | <------- Vizsgált osztály (Leszármazott)
      +-------------------+
\end{verbatim}

\subsection{Programrészlet és Szintaktika}

\begin{minted}{cpp}
// 1. A legfelső szint
class A {
public:
    int x;
};

// 2. Köztes szint
// Itt: 'A' a 'B' osztály KÖZVETLEN bázisosztálya
class B : public A {
public:
    int y;
};

// 3. Alsó szint
// Itt: 'B' a 'C' osztály KÖZVETLEN bázisosztálya
// Itt: 'A' a 'C' osztály KÖZVETETT bázisosztálya
class C : public B {
public:
    void teszt() {
        x = 10; // Eléri a közvetett ős tagját is (ha public/protected)
        y = 20; // Eléri a közvetlen ős tagját is
    }
};
\end{minted}

\subsection{Fontos szabályok}

\begin{itemize}
    \item \textbf{Konstruktor hívás:} A leszármazott osztály konstruktorának inicializáló listájában (\texttt{C() : ...}) \textbf{kizárólag a közvetlen bázisosztály} (\texttt{B}) konstruktora hívható meg. A közvetett ős (\texttt{A}) inicializálása a köztes osztály (\texttt{B}) felelőssége.
    \item \textbf{Virtuális öröklés kivétele:} Virtuális öröklés esetén (\texttt{virtual inheritance}) a „legalsó” leszármazott felelős a virtuális közvetett ős inicializálásáért.
\end{itemize}
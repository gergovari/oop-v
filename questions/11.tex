\section{Ismertesse az „öröklődés” objektum-orientált elvet!}

\subsection{Fogalma és Lényege}
\begin{itemize}
    \item \textbf{Definíció:} Az öröklődés (Inheritance) lehetővé teszi, hogy egy meglévő osztály (ős) tulajdonságait és viselkedését egy új osztály (utód) átvegye.
    \item \textbf{Cél:} A kód újrahasznosítása és a hierarchikus rendszerezés. Nem kell újra leírni a közös kódrészleteket.
    \item \textbf{"Is-a" kapcsolat:} Az öröklés "ez egy..." (is-a) kapcsolatot valósít meg. Például: Az \textit{Autó} (utód) egy \textit{Jármű} (ős).
\end{itemize}

\subsection{Terminológia}
\begin{itemize}
    \item \textbf{Ősosztály (Base class / Super class):} Az az osztály, amelynek a tulajdonságait örökítjük.
    \item \textbf{Származtatott osztály (Derived class / Sub class):} Az új osztály, amely örökli az ős tulajdonságait, és általában újakkal egészíti ki azokat.
\end{itemize}

\subsection{A \texttt{protected} (Védett) hozzáférés szerepe}
Az öröklésnél megjelenik egy harmadik láthatósági szint:
\begin{itemize}
    \item \textbf{\texttt{protected}:} A külvilág számára rejtett (mint a private), de a származtatott osztályok számára látható és módosítható.
\end{itemize}

\subsection{Szintaxis és Példa}
C++-ban az osztály neve után kettősponttal és a hozzáférési mód megadásával (általában \texttt{public}) jelöljük az öröklést.

\begin{minted}[frame=lines, framesep=2mm, baselinestretch=1.2, fontsize=\footnotesize, linenos]{cpp}
#include <iostream>

// ŐSOSZTÁLY (Szülő)
class Jarmu {
protected:
    int sebesseg; // A leszármazottak látják, a main() nem

public:
    Jarmu() : sebesseg(0) {}

    void indul() {
        std::cout << "A jarmu elindult." << std::endl;
    }
};

// SZÁRMAZTATOTT OSZTÁLY (Gyerek)
// Az Auto örökli a Jarmu minden tulajdonságát
class Auto : public Jarmu {
public:
    void gazadas() {
        // Hozzáférünk az ős 'protected' adattagjához
        sebesseg += 10; 
        std::cout << "Sebesseg: " << sebesseg << " km/h" << std::endl;
    }
    
    // Saját, új funkció, ami az ősben nem volt
    void dudal() {
        std::cout << "Tu-tu!" << std::endl;
    }
};

int main() {
    Auto kocsi;

    // 1. Az ősosztály metódusát is tudja használni
    kocsi.indul(); 

    // 2. A saját metódusait is tudja használni
    kocsi.gazadas();
    kocsi.dudal();

    return 0;
}
\end{minted}

\subsection{Összegzés}
Az \texttt{Auto} osztálynak nem kellett újra definiálnia a \texttt{sebesseg} változót vagy az \texttt{indul()} függvényt, azokat "ingyen" megkapta a \texttt{Jarmu} osztálytól, így a kód rövidebb és átláthatóbb.
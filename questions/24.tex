\section{Ismertesse a névterek definiálásának szükségességét a C++ programokban! Melyik operátorral hivatkozhatunk egy adott névtérben található osztályra?}

\subsection{A névterek (Namespaces) szükségessége}
A C++ programozásban a névterek elsődleges célja a globális névtérben fellépő zsúfoltság és a **névütközések (name collisions)** megakadályozása.

\begin{itemize}
    \item \textbf{Névütközések elkerülése:}
    Nagyobb szoftverrendszerek vagy több külső könyvtár (library) használata esetén gyakori, hogy azonos neveket használnak (pl. \texttt{Node}, \texttt{String}, \texttt{Vector}). Névterek nélkül a fordító nem tudná megkülönböztetni ezeket, ami fordítási hibát okozna.
    
    \item \textbf{Logikai szervezés:}
    A névterek lehetővé teszik a kód moduláris felépítését. A kapcsolódó osztályokat és függvényeket (pl. fájlkezelés, hálózat, grafika) logikailag elkülönített csoportokba rendezhetjük (pl. \texttt{std}, \texttt{boost}, \texttt{sfml}).
    
    \item \textbf{Globális névtér védelme:}
    Megakadályozza a globális változók és függvények véletlen felülírását vagy árnyékolását.
\end{itemize}

\subsection{Hivatkozás az elemekre}
Egy adott névtérben található osztályra vagy tagra a **hatókör-feloldó operátorral (Scope Resolution Operator)** hivatkozhatunk.

\begin{itemize}
    \item \textbf{Jele:} \texttt{::} (kettős kettőspont).
    \item \textbf{Formátum:} \texttt{NévtérNeve::Azonosító}.
    \item \textbf{Alternatíva (using):} A \texttt{using namespace ...;} utasítással a névtér elemei beemelhetők az aktuális hatókörbe, így az operátor elhagyható, de ez a névütközések veszélye miatt óvatosan használandó.
\end{itemize}

\subsection{Példakód}
Az alábbi példa két azonos nevű osztály (\texttt{Connection}) békés együttélését mutatja be névterek segítségével.

\begin{minted}[frame=lines, framesep=2mm, baselinestretch=1.2, fontsize=\footnotesize, linenos]{cpp}
#include <iostream>

// 1. Hálózati modul névtere
namespace Network {
    class Connection {
    public:
        void connect() { std::cout << "Connecting via TCP..." << std::endl; }
    };
}

// 2. Adatbázis modul névtere
namespace Database {
    class Connection {
    public:
        void connect() { std::cout << "Connecting to SQL..." << std::endl; }
    };
}

int main() {
    // Névtér minősítés nélkül fordítási hiba lenne:
    // Connection c; // HIBA: "ambiguous"
    
    // Használat a :: operátorral
    Network::Connection netConn;
    Database::Connection dbConn;
    
    netConn.connect();
    dbConn.connect();
    
    // Using deklaráció egy adott elemre
    using Network::Connection;
    Connection c; // Most a Network::Connection-t jelenti
    
    return 0;
}
\end{minted}
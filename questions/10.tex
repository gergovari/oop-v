\section{Ismertesse az „adatrejtés” objektum-orientált elvet!}

\subsection{Fogalma és Lényege}
\begin{itemize}
    \item \textbf{Definíció:} Az adatrejtés (Data Hiding) az a technika, amellyel az objektum belső állapotát (adattagjait) elzárjuk a külvilág elől.
    \item \textbf{Cél:} Megakadályozni, hogy más programrészek véletlenül vagy szándékosan, ellenőrizetlenül módosítsák az objektum adatait.
    \item \textbf{Interfész és Implementáció szétválasztása:}
    \begin{itemize}
        \item \textbf{Implementáció (Privát):} A belső működés részletei és az adatok tárolása. Ez kívülről láthatatlan.
        \item \textbf{Interfész (Publikus):} Azok a metódusok (függvények), amelyeken keresztül kommunikálni lehet az objektummal.
    \end{itemize}
\end{itemize}

\subsection{Megvalósítás C++ nyelven}
A hozzáférési szinteket (access specifiers) használjuk a láthatóság szabályozására:

\begin{enumerate}
    \item \textbf{\texttt{private} (Privát):}
    \begin{itemize}
        \item Az osztály alapértelmezett hozzáférése (class esetén).
        \item Csak az osztály saját metódusai férnek hozzá.
        \item \textbf{Ide helyezzük az adattagokat.}
    \end{itemize}
    \item \textbf{\texttt{public} (Nyilvános):}
    \begin{itemize}
        \item Bárhonnan elérhető.
        \item \textbf{Ide helyezzük a Getter/Setter metódusokat}, amelyek ellenőrzött hozzáférést biztosítanak a privát adatokhoz.
    \end{itemize}
\end{enumerate}

\subsection{Miért fontos? (Előnyök)}
\begin{itemize}
    \item \textbf{Érvényesség (Validáció):} A Setter metódusokban megvizsgálhatjuk a kapott értéket. Ha érvénytelen (pl. negatív életkor), megakadályozhatjuk a beállítást.
    \item \textbf{Karbantarthatóság:} Ha megváltoztatjuk az adattárolás belső módját (pl. \texttt{int} helyett \texttt{long} vagy adatbázisból jön), a külvilágnak nem kell erről tudnia, amíg a publikus metódusok ugyanúgy hívhatók.
    \item \textbf{Csak olvashatóság:} Ha egy adathoz írunk Gettert, de Settert nem, akkor az adat kívülről "read-only" (csak olvasható) lesz.
\end{itemize}

\subsection{Példa: Ellenőrzött hozzáférés}
A példában a `Diak` osztály osztályzatait rejtjük el. Csak 1 és 5 közötti értéket engedünk beállítani.

\begin{minted}[frame=lines, framesep=2mm, baselinestretch=1.2, fontsize=\footnotesize, linenos]{cpp}
class Diak {
private:
    // REJTETT ADAT: Kívülről közvetlenül nem érhető el.
    int jegy; 

public:
    // Konstruktor
    Diak() : jegy(1) {}

    // SETTER (Beállító) - Validációval
    // Ez az "ajtó", amin keresztül módosítani lehet az adatot.
    void setJegy(int ujJegy) {
        if (ujJegy >= 1 && ujJegy <= 5) {
            jegy = ujJegy;
        } else {
            // Érvénytelen adat elutasítása
            // (Itt lehetne hibaüzenetet dobni vagy logolni)
        }
    }

    // GETTER (Lekérdező)
    // Ez biztosítja, hogy az adat olvasható legyen.
    int getJegy() const {
        return jegy;
    }
};

int main() {
    Diak d;
    
    // d.jegy = 6;  // -> FORDÍTÁSI HIBA: 'jegy' private!

    d.setJegy(4);   // Működik: érvényes adat
    d.setJegy(8);   // Nem történik semmi: érvénytelen adat, a védelem működik
    
    return 0;
}
\end{minted}
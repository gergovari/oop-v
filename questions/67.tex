\section{Ismertesse az aggregációs operátorok definícióját, és 5 axiómáját!}

A fuzzy rendszerekben és döntéstámogatásban az aggregációs operátorok feladata, hogy több bemeneti értékből (pl. több szabály következtetéséből vagy több kritérium értékeléséből) egyetlen összevont értéket állítsanak elő.

\subsection{Definíció}

Az $h: [0, 1]^n \to [0, 1]$ leképezést aggregációs operátornak nevezzük, ha az több ($n$ darab), a $[0, 1]$ zárt intervallumba eső bemenethez rendel hozzá egyetlen, szintén a $[0, 1]$ intervallumba eső kimeneti értéket.

Formálisan: $y = h(x_1, x_2, \dots, x_n)$.

\subsection{Az 5 alapvető axióma (Tulajdonság)}

A szakirodalom általában az alábbi öt tulajdonságot (axiómát) várja el egy általános célú, jól viselkedő aggregációs operátortól:

\begin{enumerate}
    \item \textbf{Peremfeltételek (Boundary Conditions):}
        Az operátornak meg kell őriznie az univerzum szélsőértékeit. Ha minden bemenet a minimum, az eredmény is minimum; ha minden bemenet maximum, az eredmény is maximum.
        $$ h(0, 0, \dots, 0) = 0 $$
        $$ h(1, 1, \dots, 1) = 1 $$

    \item \textbf{Monotonitás (Monotonicity):}
        Az operátor nem csökkenő függvény a változói szerint. Ha bármelyik bemeneti érték növekszik (miközben a többi változatlan), a kimeneti érték nem csökkenhet.
        $$ \text{Ha } x_i \leq y_i \text{ minden } i\text{-re, akkor } h(\mathbf{x}) \leq h(\mathbf{y}) $$

    \item \textbf{Folytonosság (Continuity):}
        Az operátor legyen folytonos függvény a teljes értelmezési tartományon. Ez biztosítja a rendszer stabilitását: a bemenetek kis megváltozása csak kis változást okozhat a kimeneten, nincsenek ugrásszerű változások.

    \item \textbf{Szimmetria vagy Kommutativitás (Symmetry):}
        A bemeneti változók sorrendje nem befolyásolhatja az eredményt (a bemenetek egyenrangúak).
        $$ h(x_1, x_2, \dots, x_n) = h(x_{\pi(1)}, x_{\pi(2)}, \dots, x_{\pi(n)}) $$
        ahol $\pi$ az indexek egy tetszőleges permutációja.

    \item \textbf{Idempotencia (Idempotency):}
        Ha az összes bemenet ugyanazt a $c$ értéket veszi fel, akkor az aggregált értéknek is pontosan $c$-nek kell lennie.
        $$ h(c, c, \dots, c) = c $$
        \textit{Megjegyzés:} Ez a tulajdonság jellemzően az átlagoló (averaging) operátorokra igaz, a t-normákra és s-normákra általában nem.
\end{enumerate}

\subsection{Osztályozás az idempotencia alapján}

Az 5. axiómához való viszony alapján az operátorok lehetnek:
\begin{itemize}
    \item \textbf{Átlagoló (Averaging):} Teljesítik az idempotenciát (pl. számtani közép).
    \item \textbf{Konjunktív (And-like):} $h(c, \dots, c) \leq c$ (pl. t-normák, min).
    \item \textbf{Diszjunktív (Or-like):} $h(c, \dots, c) \geq c$ (pl. s-normák, max).
\end{itemize}
\section{Ismertesse az osztályok egyoperandusú műveleteinek átdefiniálási lehetőségeit! Írjon példákat minden egyes lehetőséghez!}

\subsection{Áttekintés}
Az egyoperandusú (unáris) operátorok (pl. \texttt{++}, \texttt{--}, \texttt{-} (negálás), \texttt{!}) egyetlen objektumon fejtik ki hatásukat. Ezen operátorok túlterhelésére két fő lehetőség van, illetve egy speciális szabály vonatkozik a prefix/postfix megkülönböztetésre.

\subsection{1. Lehetőség: Tagfüggvényként (Member Function)}
Ha az operátort az osztály tagjaként definiáljuk:
\begin{itemize}
    \item \textbf{Paraméterek száma:} Általában \textbf{0 paramétere} van.
    \item \textbf{Operandus:} Az operátor implicit módon az aktuális objektumon (\texttt{*this}) hajtódik végre.
    \item \textbf{Előnye:} Közvetlenül hozzáfér a privát adattagokhoz, nem kell \texttt{friend} deklaráció.
\end{itemize}

\subsection{2. Lehetőség: Globális (Barát) függvényként (Global Function)}
Ha az operátort az osztályon kívül definiáljuk:
\begin{itemize}
    \item \textbf{Paraméterek száma:} \textbf{1 paramétere} van (az osztály típusú objektum referenciája).
    \item \textbf{Operandus:} A függvény paraméterként kapja meg az objektumot.
    \item \textbf{Használat:} Gyakran \texttt{friend}-ként deklarálják az osztályban a hozzáférés miatt.
\end{itemize}

\subsection{Speciális eset: Prefix vs. Postfix (++ és --)}
Mivel a \texttt{++a} (prefix) és \texttt{a++} (postfix) operátorok ugyanazt a jelet használják, a C++ egy mesterséges paraméterrel különbözteti meg őket:
\begin{itemize}
    \item \textbf{Prefix (Előtag):} Nincs paraméter (vagy globálisnál 1 db referencia). Referenciával tér vissza a módosított objektumra.
    \item \textbf{Postfix (Utótag):} Egy \textbf{fiktív \texttt{int} paramétert} kap. Értékkel tér vissza (a módosítás előtti állapottal).
\end{itemize}

\subsection{Példakód}
Az alábbi példa bemutatja a három leggyakoribb esetet: a negálást globális függvényként, valamint a prefix és postfix inkrementálást tagfüggvényként.

\begin{minted}[frame=lines, framesep=2mm, baselinestretch=1.2, fontsize=\footnotesize, linenos]{cpp}
#include <iostream>

class Szam {
private:
    int ertek;

public:
    Szam(int v = 0) : ertek(v) {}

    // 1. ESET: Prefix inkrementálás (++obj) TAGFÜGGVÉNYKÉNT
    // Nincs paraméter.
    // Előbb növelünk, aztán visszaadjuk önmagát referenciaként.
    Szam& operator++() {
        this->ertek += 1;
        return *this;
    }

    // 2. ESET: Postfix inkrementálás (obj++) TAGFÜGGVÉNYKÉNT
    // A "dummy" int paraméter jelzi a fordítónak, hogy ez postfix.
    // Elmentjük a régit, növelünk, visszaadjuk a régit érték szerint.
    Szam operator++(int) {
        Szam regi = *this; // Másolat készítése
        this->ertek += 1;  // Növelés
        return regi;       // Régi érték visszaadása
    }

    // Kiíratáshoz
    void print() const { std::cout << "Ertek: " << ertek << std::endl; }

    // Barát deklaráció a globális negáláshoz
    friend Szam operator-(const Szam& sz);
};

// 3. ESET: Negálás (-obj) GLOBÁLIS FÜGGVÉNYKÉNT
// Egy paramétert kap. Új objektumot ad vissza.
Szam operator-(const Szam& sz) {
    return Szam(-sz.ertek);
}

int main() {
    Szam n(10);

    ++n;       // Prefix hívás (most 11)
    n.print();

    n++;       // Postfix hívás (most 12, de a kifejezés értéke 11 volt)
    n.print();

    Szam neg = -n; // Globális unáris mínusz hívása (-12)
    neg.print();

    return 0;
}
\end{minted}
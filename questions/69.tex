\section{Ismertesse az OWA operátort!}

Az OWA (Ordered Weighted Averaging – Rendezett Súlyozott Átlag) operátort Ronald R. Yager vezette be 1988-ban. Ez egy olyan aggregációs művelet, amely hidat képez a \textbf{Minimum} (ÉS jellegű) és a \textbf{Maximum} (VAGY jellegű) operátorok között, lehetővé téve a „szigorúság” finomhangolását.

\subsection{Definíció}

Az OWA operátor egy $F: \pmb{R}^n \to \pmb{R}$ leképezés, amelyhez tartozik egy $W = [w_1, w_2, \dots, w_n]$ súlyvektor, ahol:
\begin{enumerate}
    \item $w_i \in [0, 1]$
    \item $\sum_{i=1}^{n} w_i = 1$
\end{enumerate}

A függvény értéke:
$$ F(a_1, a_2, \dots, a_n) = \sum_{j=1}^{n} w_j b_j $$

Ahol $b_j$ az input argumentumok ($a_1, \dots, a_n$) \textbf{csökkenő sorrendbe rendezett} permutációjának $j$-edik eleme.
$$ b_1 \ge b_2 \ge \dots \ge b_n $$

\subsection{A legfontosabb különbség a Súlyozott Átlaghoz képest}

A hagyományos súlyozott átlagnál ($WA$) a súlyok a konkrét \textit{adatforráshoz} (attribútumhoz) tartoznak. Az OWA operátornál a súlyok a \textbf{pozícióhoz (rangsorhoz)} tartoznak.

\begin{itemize}
    \item \textbf{WA:} A $w_1$ súly mindig az $a_1$ bemenetet szorozza, függetlenül annak értékétől.
    \item \textbf{OWA:} A $w_1$ súly mindig a \textbf{legnagyobb} bemeneti értéket szorozza, a $w_2$ a második legnagyobbat, és így tovább.
\end{itemize}

\subsection{Speciális esetek (A súlyvektor függvényében)}

A súlyvektor beállításával az operátor viselkedése változik a két szélsőség között:

\begin{itemize}
    \item \textbf{Maximum (VAGY):} $W^* = [1, 0, \dots, 0]$
        \begin{itemize}
            \item Csak a legnagyobb elemet veszi figyelembe.
        \end{itemize}
    \item \textbf{Minimum (ÉS):} $W_* = [0, 0, \dots, 1]$
        \begin{itemize}
            \item Csak a legkisebb elemet veszi figyelembe.
        \end{itemize}
    \item \textbf{Számtani közép:} $W_{avg} = [\frac{1}{n}, \frac{1}{n}, \dots, \frac{1}{n}]$
        \begin{itemize}
            \item Minden elemet egyformán súlyoz.
        \end{itemize}
    \item \textbf{Olimpiai átlag:} A szélsőértékek kizárása (pl. az első és utolsó súly 0, a többi egyenlő).
\end{itemize}

\subsection{Jellemzők: Orness és Diszperzió}

Az operátor karakterisztikáját két mérőszámmal írhatjuk le:

\begin{enumerate}
    \item \textbf{Orness (VAGY-szerűség):} Azt méri, mennyire hasonlít az operátor a Maximumhoz (optimista viselkedés).
        \begin{itemize}
            \item Ha a súlyok az indexek elején tömörülnek $\rightarrow$ Magas Orness.
            \item Ha a súlyok az indexek végén tömörülnek $\rightarrow$ Alacsony Orness (ÉS-szerű).
        \end{itemize}
    \item \textbf{Diszperzió (Entrópia):} Azt méri, mennyire használja ki az összes információt.
        \begin{itemize}
            \item Ha $W = [\frac{1}{n}, \dots]$, a diszperzió maximális.
            \item Ha $W = [1, 0, \dots]$, a diszperzió 0.
        \end{itemize}
\end{enumerate}

\subsection{Számítási Példa}

Legyenek a bemeneti értékek (pl. három bíráló pontszáma): $A = [0.4, \mathbf{0.9}, 0.7]$.
Legyen a súlyvektor (optimista hozzáállás): $W = [0.6, 0.3, 0.1]$.

\begin{minted}{python}
# 1. Bemenetek
inputs = [0.4, 0.9, 0.7]

# 2. RENDEZÉS (Ez a kritikus lépés!)
# Csökkenő sorrendbe állítjuk az értékeket:
sorted_inputs = [0.9, 0.7, 0.4] 
# (b1=0.9, b2=0.7, b3=0.4)

# 3. Súlyozott összegzés
# w1*b1 + w2*b2 + w3*b3
result = (0.6 * 0.9) + (0.3 * 0.7) + (0.1 * 0.4)

# Számítás: 0.54 + 0.21 + 0.04
# OWA Eredmény = 0.79
\end{minted}

Látható, hogy az eredmény (0.79) magasabb, mint a számtani átlag ($\approx 0.66$), mert a legnagyobb érték kapta a legnagyobb súlyt.
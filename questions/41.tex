\section{Ismertesse az ősben található osztálytagok elérésének módosítását private és protected öröklődés során!}

C++-ban az öröklődés típusa (hozzáférési módosítója) szabályozza, hogy az ősosztály tagjai milyen láthatósággal jelenjenek meg a származtatott osztályban. Általános szabály, hogy az öröklés soha nem tágítja, csak szűkítheti (vagy szinten tarthatja) a láthatóságot.

\subsection{Az alapvető mechanizmus}

A végső elérési szintet a „legszigorúbb szabály” elve határozza meg:
$$ \text{Új láthatóság} = \max(\text{Eredeti láthatóság}, \text{Öröklés típusa}) $$
Ahol a szigorúsági sorrend: \texttt{private} > \texttt{protected} > \texttt{public}.
Az ős \texttt{private} tagjai soha nem érhetők el közvetlenül a származtatott osztályból, függetlenül az öröklés típusától.

\subsection{Protected (Védett) öröklődés}

Ha \texttt{class Derived : protected Base} formában származtatunk:

\begin{itemize}
    \item \textbf{Hatása:}
        \begin{itemize}
            \item Az ős \textbf{public} tagjai $\rightarrow$ \textbf{protected} tagokká válnak az utódban.
            \item Az ős \textbf{protected} tagjai $\rightarrow$ \textbf{protected} tagok maradnak.
            \item Az ős \textbf{private} tagjai $\rightarrow$ elérhetetlenek maradnak.
        \end{itemize}
    \item \textbf{Következmény:}
        \begin{itemize}
            \item A külvilág (pl. \texttt{main} függvény) számára az összes örökölt tag elérhetetlenné válik (mivel védettek lettek).
            \item A további leszármazottak (az unokák) viszont még hozzáférhetnek ezekhez a tagokhoz (mivel protected státuszúak).
        \end{itemize}
\end{itemize}

\subsection{Private (Privát) öröklődés}

Ha \texttt{class Derived : private Base} formában származtatunk (vagy elhagyjuk a kulcsszót \texttt{class} esetén):

\begin{itemize}
    \item \textbf{Hatása:}
        \begin{itemize}
            \item Az ős \textbf{public} tagjai $\rightarrow$ \textbf{private} tagokká válnak az utódban.
            \item Az ős \textbf{protected} tagjai $\rightarrow$ \textbf{private} tagokká válnak az utódban.
            \item Az ős \textbf{private} tagjai $\rightarrow$ elérhetetlenek maradnak.
        \end{itemize}
    \item \textbf{Következmény:}
        \begin{itemize}
            \item A külvilág számára minden el van rejtve.
            \item A további leszármazottak (unokák) már semmit sem látnak az ősosztályból (mivel itt a lánc megszakad a priváttá tétel miatt).
            \item Ez gyakorlatilag implementációs öröklés (hasonló a kompozícióhoz).
        \end{itemize}
\end{itemize}

\subsection{Összefoglaló táblázat}

\begin{table}[h!]
    \centering
    \begin{tabular}{|l|l|l|l|}
    \hline
    \textbf{Ős tagja} & \textbf{Public öröklés} & \textbf{Protected öröklés} & \textbf{Private öröklés} \\ \hline
    \texttt{public} & \texttt{public} & \texttt{protected} & \texttt{private} \\ \hline
    \texttt{protected} & \texttt{protected} & \texttt{protected} & \texttt{private} \\ \hline
    \texttt{private} & \textit{Elérhetetlen} & \textit{Elérhetetlen} & \textit{Elérhetetlen} \\ \hline
    \end{tabular}
\end{table}

\subsection{Láthatóság visszaállítása (Using declaration)}

Privát vagy védett öröklés esetén is visszaállítható egyes tagok láthatósága a \texttt{public} szintre a \texttt{using} kulcsszóval.

\begin{minted}{cpp}
class Base {
public:
    void fgv() {}
    int adat;
};

class Derived : private Base {
public:
    // Kivétel: ezt az egy tagot publikussá tesszük
    using Base::fgv; 
    // Az 'adat' továbbra is private marad
};
\end{minted}

\subsection{Példa a korlátozásokra}

\begin{minted}{cpp}
class Base {
public:    int pub;
protected: int prot;
private:   int priv;
};

// 1. Protected öröklés
class ProtDerived : protected Base {
    void teszt() {
        pub = 1;  // OK (itt protected)
        prot = 2; // OK (itt protected)
        // priv = 3; // HIBA: ős privátja nem érhető el
    }
};

// 2. Private öröklés
class PrivDerived : private Base {
    void teszt() {
        pub = 1;  // OK (itt private)
        prot = 2; // OK (itt private)
    }
};

class Unoka : public PrivDerived {
    void teszt() {
        // pub = 1; // HIBA! A PrivDerived-ben ez már private lett!
    }
};

int main() {
    ProtDerived pd;
    // pd.pub = 1; // HIBA! Kívülről protected, nem látszik.
    return 0;
}
\end{minted}
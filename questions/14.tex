\section{Ismertesse a „private”, „protected”, „public” módosítók működését az osztálytagok definiálásakor!}

\subsection{A hozzáférési szintek célja}
A C++ nyelvben a hozzáférési módosítók (access specifiers) szabályozzák az egységbe zárást (encapsulation). Azt határozzák meg, hogy az osztály adattagjaihoz és tagfüggvényeihez a kód mely részeiből lehet hozzáférni.

\subsection{A három módosító részletezése}

\begin{enumerate}
    \item \textbf{\texttt{public} (Nyilvános):}
    \begin{itemize}
        \item \textbf{Láthatóság:} Bárhonnan elérhető (az osztályon belülről, leszármazott osztályokból és a külvilágból, pl. a \texttt{main}-ből is).
        \item \textbf{Használat:} Az osztály publikus interfésze (azok a függvények, amelyeket a felhasználónak szánunk).
    \end{itemize}

    \item \textbf{\texttt{protected} (Védett):}
    \begin{itemize}
        \item \textbf{Láthatóság:} A külvilág számára rejtett, de az osztály saját metódusai és a \textbf{leszármazott osztályok} (gyerekosztályok) hozzáférhetnek.
        \item \textbf{Használat:} Olyan segédváltozók vagy függvények, amelyek a belső működéshez kellenek, és szeretnénk megosztani az öröklési hierarchiában.
    \end{itemize}

    \item \textbf{\texttt{private} (Privát):}
    \begin{itemize}
        \item \textbf{Láthatóság:} Kizárólag az adott osztály saját metódusai (és a \texttt{friend} osztályok) láthatják. Még a leszármazottak sem férnek hozzá!
        \item \textbf{Használat:} Adattagok (belső állapot) védelme. Ez az alapértelmezett láthatóság \texttt{class} esetén.
    \end{itemize}
\end{enumerate}

\subsection{Összehasonlító táblázat}
\begin{center}
\begin{tabular}{|l|c|c|c|}
\hline
\textbf{Módosító} & \textbf{Saját osztály} & \textbf{Leszármazott} & \textbf{Külvilág (pl. main)} \\
\hline
\texttt{public} & IGEN & IGEN & IGEN \\
\hline
\texttt{protected} & IGEN & IGEN & \textbf{NEM} \\
\hline
\texttt{private} & IGEN & \textbf{NEM} & \textbf{NEM} \\
\hline
\end{tabular}
\end{center}

\subsection{Demonstrációs példa}

\begin{minted}[frame=lines, framesep=2mm, baselinestretch=1.2, fontsize=\footnotesize, linenos]{cpp}
class OsOsztaly {
public:
    int publikusAdat;    // Mindenki látja
protected:
    int vedettAdat;      // Csak a család (leszármazottak) látja
private:
    int privatAdat;      // Titok (csak ez az osztály látja)

public:
    void teszt() {
        privatAdat = 1;  // OK: Belül vagyunk
    }
};

class Leszarmazott : public OsOsztaly {
public:
    void hozzaferesTeszt() {
        publikusAdat = 2; // OK
        vedettAdat = 3;   // OK: Örököltük, láthatjuk
        
        // privatAdat = 4; // -> HIBA: A gyerek sem láthatja a szülő privátját!
    }
};

int main() {
    OsOsztaly obj;
    
    obj.publikusAdat = 10; // OK
    
    // obj.vedettAdat = 20; // -> HIBA: Kívülről ez olyan, mintha private lenne
    // obj.privatAdat = 30; // -> HIBA: Nem látható
    
    return 0;
}
\end{minted}
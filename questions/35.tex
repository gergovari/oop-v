\section{Ismertesse a fájlok kezelésére használt osztályt, gyakran használt metódusait és operátorait!}

A C++ nyelvben a fájlkezelés a streameken (adatfolyamokon) keresztül történik, hasonlóan a konzolos kommunikációhoz. A szükséges osztályokat az \texttt{<fstream>} header tartalmazza.

\subsection{Az alapvető osztályok}

A fájlkezelés iránya határozza meg, melyik osztályt használjuk:

\begin{itemize}
    \item \textbf{\texttt{std::ofstream} (Output File Stream):} Fájlba írásra szolgál. Ha a fájl nem létezik, létrehozza.
    \item \textbf{\texttt{std::ifstream} (Input File Stream):} Fájlból való olvasásra szolgál.
    \item \textbf{\texttt{std::fstream} (File Stream):} Kétirányú kommunikációt (írást és olvasást is) lehetővé tesz.
\end{itemize}

\subsection{Megnyitás és fájlmódok}

A fájlokat megnyithatjuk a konstruktorban vagy az \texttt{open()} metódussal. A második paraméter határozza meg a megnyitás módját (ezek kombinálhatók a \texttt{|} operátorral):

\begin{itemize}
    \item \textbf{\texttt{std::ios::in}:} Megnyitás olvasásra (alapértelmezett \texttt{ifstream}-nél).
    \item \textbf{\texttt{std::ios::out}:} Megnyitás írásra (alapértelmezett \texttt{ofstream}-nél). Felülírja a fájlt!
    \item \textbf{\texttt{std::ios::app}:} Hozzáfűzés (Append). A meglévő tartalom megmarad, az írás a végére kerül.
    \item \textbf{\texttt{std::ios::binary}:} Bináris mód (pl. képek, struktúrák mentésekor), kikapcsolja a szöveges konverziókat.
\end{itemize}

\subsection{Fontos tagfüggvények (Metódusok)}

\begin{itemize}
    \item \textbf{\texttt{open(filename, mode)}:} Hozzárendeli a streamet egy fizikai fájlhoz.
    \item \textbf{\texttt{is\_open()}:} Logikai értékkel tér vissza: sikerült-e a fájl megnyitása? (Mindig ellenőrizni kell!).
    \item \textbf{\texttt{close()}:} Bezárja a fájlt és menti a pufferek tartalmát. (A destruktor is meghívja, de ajánlott explicit módon használni).
    \item \textbf{\texttt{eof()}:} (End Of File) Igazat ad, ha elértük a fájl végét olvasáskor.
    \item \textbf{\texttt{getline(stream, string)}:} Globális segédfüggvény, amely egy teljes sort olvas be a fájlból (a szóközt is beleértve), amíg sortörést nem talál.
\end{itemize}

\subsection{Operátorok}

Mivel az osztályok az \texttt{iostream}-ből származnak, az operátorok megegyeznek a konzolos I/O-val:

\begin{itemize}
    \item \textbf{\texttt{<<} (Insertion):} Adat írása a fájlba (formázott szövegként).
    \item \textbf{\texttt{>>} (Extraction):} Adat olvasása a fájlból (whitespace karaktereknél megáll).
\end{itemize}

\subsection{Példa: Írás és olvasás}

\begin{minted}{cpp}
#include <iostream>
#include <fstream>
#include <string>

int main() {
    // 1. Írás fájlba
    std::ofstream kimenet("adatok.txt"); // Létrehozás és megnyitás
    if (kimenet.is_open()) {
        kimenet << "Elso sor" << std::endl;
        kimenet << 123 << std::endl;
        kimenet.close(); // Lezárás
    }

    // 2. Olvasás fájlból
    std::ifstream bemenet("adatok.txt");
    std::string sor;

    if (bemenet.is_open()) {
        // Soronkénti beolvasás while ciklussal
        // A getline visszatérési értéke maga a stream, ami false-t ad hiba/EOF esetén
        while (std::getline(bemenet, sor)) {
            std::cout << "Beolvasva: " << sor << std::endl;
        }
        bemenet.close();
    } else {
        std::cerr << "Hiba a fajl megnyitasakor!" << std::endl;
    }

    return 0;
}
\end{minted}
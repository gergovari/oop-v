\section{Ismertesse példával a „kompozíció” elvet osztályok egymásba ágyazására!}

A kompozíció (összetétel) az objektumorientált programozás egyik alapvető építőköve, amely a „tartalmazás” (vagy „has-a”, azaz „van neki”) kapcsolatot valósítja meg két osztály között.

\subsection{A kompozíció elvei}

\begin{itemize}
    \item \textbf{„Has-a” kapcsolat:} Az öröklődéssel ellentétben (ami „is-a” típusú), itt az egyik objektum birtokolja a másikat (pl. a Számítógépnek \textit{van} Processzora).
    \item \textbf{Szoros csatolás (Strong association):} A tartalmazott objektum (rész) élettartama függ a tartalmazó objektum (egész) élettartamától. Ha az „egész” megszűnik, a „rész” is megsemmisül.
    \item \textbf{Újrafelhasználhatóság:} Lehetővé teszi bonyolult objektumok felépítését egyszerűbb, már meglévő osztályokból.
\end{itemize}

\subsection{Megvalósítás C++ nyelven}

Technikailag a kompozíciót úgy valósítjuk meg, hogy egy osztály típusú változót adattagként deklarálunk egy másik osztályban.

\begin{itemize}
    \item \textbf{Adattag:} Az objektumot érték szerint tároljuk (nem mutatóként), így a memóriakezelést a fordító automatikusan végzi.
    \item \textbf{Member Initializer List:} A tartalmazott objektum konstruktorát a tartalmazó osztály konstruktorának inicializáló listájában kell meghívni. Ez kritikus fontosságú, ha a belső objektumnak nincs paraméter nélküli (default) konstruktora.
\end{itemize}

\subsection{Példa: Számítógép és Processzor}

Az alábbi példában a \texttt{Szamitogep} osztály kompozícióval tartalmazza a \texttt{CPU} osztályt. A CPU inicializálása a Számítógép létrehozásakor történik.

\begin{minted}{cpp}
#include <iostream>

// A "rész" osztály
class CPU {
    int frekvencia;
public:
    // Nincs default konstruktor, paraméter kötelező
    CPU(int freq) : frekvencia(freq) {
        std::cout << "CPU beepitve: " << frekvencia << " MHz" << std::endl;
    }

    void dolgozik() {
        std::cout << "CPU szamol..." << std::endl;
    }
};

// Az "egész" osztály
class Szamitogep {
private:
    // Kompozíció: A CPU a Szamitogep adattagja
    CPU processzor; 

public:
    // Konstruktor
    // A "processzor" adattagot az inicializáló listán KELL beállítani,
    // mielőtt a konstruktor törzse lefutna.
    Szamitogep(int freq) : processzor(freq) {
        std::cout << "Szamitogep bekapcsolva." << std::endl;
    }

    void futtat() {
        // A feladatot delegáljuk a belső objektumnak
        processzor.dolgozik(); 
    }
}; 
// Destruktor lefutásakor: először a Szamitogep szűnik meg, 
// majd automatikusan a processzor is.

int main() {
    Szamitogep pc(3200);
    pc.futtat();
    return 0;
}
\end{minted}
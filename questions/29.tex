\section{Ismertesse a „()” operátor túlterhelési lehetőségeit!}

\subsection{Áttekintés és Tulajdonságok}
A függvényhívás operátor \texttt{operator()} túlterhelésével hozhatjuk létre az úgynevezett \textbf{funktorokat} (function objects). Ez lehetővé teszi, hogy egy objektumpéldányt úgy használjunk, mintha az egy függvény lenne.

\begin{itemize}
    \item \textbf{Kizárólag tagfüggvényként:} A \texttt{()} operátor csak nem-statikus tagfüggvényként definiálható, globális függvényként nem.
    \item \textbf{Tetszőleges paraméterszám:} Ez az egyetlen operátor a C++-ban, amely tetszőleges számú ($0, 1, 2, \dots, n$) paramétert fogadhat. Emiatt gyakran használják többdimenziós tömbök (pl. mátrixok) indexelésére, mivel a \texttt{[]} operátor (hagyományosan) csak egy paramétert fogadhat.
    \item \textbf{Állapotmegőrzés (Stateful):} A hagyományos függvényekkel ellentétben a funktorok rendelkezhetnek belső állapottal (adattagokkal), amelyek megőrződnek a hívások között.
    \item \textbf{Túlterhelhetőség:} Egy osztályon belül többször is definiálható eltérő paraméterlistával (overloading).
\end{itemize}

\subsection{Gyakori felhasználási területek}
\begin{enumerate}
    \item \textbf{Paraméterezhető műveletek:} Olyan „függvények” létrehozása, amelyek viselkedése konstruktorban állítható be (pl. egy számláló vagy egy küszöbérték-vizsgáló).
    \item \textbf{STL algoritmusok:} A \texttt{std::sort}, \texttt{std::for\_each} és hasonló algoritmusok gyakran várnak funktorokat predikátumként.
    \item \textbf{Mátrix-kezelés:} \texttt{matrix(sor, oszlop)} formátumú elérés biztosítása.
\end{enumerate}

\subsection{Példakód}
Az alábbi példa egy lineáris transzformációt ($y = ax + b$) megvalósító funktort mutat be, ahol az $a$ és $b$ paraméterek az objektum állapotát képezik.

\begin{minted}[frame=lines, framesep=2mm, baselinestretch=1.2, fontsize=\footnotesize, linenos]{cpp}
#include <iostream>

class LinearTransform {
private:
    // Belső állapot (State)
    double slope;     // a
    double intercept; // b

public:
    // Konstruktor: beállítja a működési paramétereket
    LinearTransform(double a, double b) : slope(a), intercept(b) {}

    // Az operátor túlterhelése
    // Tetszőleges visszatérési érték és paraméterezés lehetséges
    double operator()(double x) const {
        return (slope * x) + intercept;
    }
};

class Matrix {
    int data[10][10];
public:
    // Példa több paraméteres használatra (Mátrix indexelés)
    // A [] operátorral ezt nem lehetne így (több paraméterrel) megoldani.
    int& operator()(int row, int col) {
        return data[row][col];
    }
};

int main() {
    // 1. Funktor példányosítása (a=2, b=3)
    LinearTransform func(2.0, 3.0);

    // 2. Használat függvényhívás szintaxissal
    // A fordító ezt hívja: func.operator()(5.0)
    double result = func(5.0); // 2 * 5 + 3 = 13

    std::cout << "Eredmény: " << result << std::endl;

    // 3. Mátrix példa
    Matrix m;
    m(1, 2) = 42; // Írás a (1,2) pozícióra

    return 0;
}
\end{minted}